\documentclass[11pt]{article}
\usepackage{amsmath}
\usepackage{fontspec}
\usepackage{listings}
\usepackage{fullpage}
\setmainfont[Ligatures=TeX]{Linux Libertine O}
\begin{document}

\section{Wstęp}
Znając dokładne odległości od każdego z trzech punktów możemy analitycznie
wyprowadzić referencyjne rozwiązanie, rozwiązując układ trzech równań okręgu:
\[
    \begin{cases}
        (x - x_A)^2 + y^2 &= d_A^2 \\
        x^2 + y^2 &= d_C^2 \\
        x^2 + (y-y_B)^2 &= d_B^2 \\
    \end{cases}
\]
Korzystając z pierwszych dwóch równań mamy
\[ x = \frac{x_A^2 + d_C^2 - d_A^2}{2x_A} \]
\[ y = \pm \sqrt{d_C^2 - x^2} \]
Dla rozwiązywanego problemu mamy
\[ x_A = 50 \]
\[ y_B = 100 \]
\[ d_A = 318.8636072053379\mbox{km} \]
\[ d_B = 249.9479945908749\mbox{km} \]
\[ d_C = 336.8590209568388\mbox{km} \]
a więc 
\[ x = 143 \]
\[ y = 305 \]

\includegraphics[scale=.7]{circles.eps}

System nawigacji hiperbolicznej taki jak LORAN pozwala rozwiązać tego rodzaju
problem zakładając znajomość jedynie różnicy odległości pomiędzy stacjami
nadawczymi.  Takie założenie czynię również przy dalszych rozważaniach.


\section{Wyznaczenie hiperbol}
\subsection{Punkty A i C}
Rozważmy kanoniczną postać równania hiperboli
\[ \frac{(x-x_0)^2}{a^2} - \frac{(y-y_0)^2}{b^2} = 1 \]
W przypadku rozważanego problemu $x_0 = \frac{x_A}{2}$ i $y_0 = 0$.

Niech $c$ - odległość pomiędzy środkiem $(x_0, y_0)$ hiperboli i każdym z ognisk
A i C.  Wtedy musi zachodzić
\[ a^2 + b^2 = c^2 \]
Dla naszego problemu mamy
\[ c = \frac{x_A}{2} \]
Równania te wyznaczają rodzinę hiperbol dla ognisk A i C.  Znając dodatkowo
różnicę odległości szukanego punktu i każdego z ognisk możemy wyznaczyć równanie
hiperboli zawierającej ten punkt korzystając z faktu, że
\[ 2a = |d_A - d_C| \]

\includegraphics[scale=.7]{hyperbolas_ew.eps}

\end{document}
